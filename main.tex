\documentclass[12pt]{article}
\usepackage{graphicx} % Required for inserting images
\usepackage{fancyhdr}
\usepackage{hyperref}

\pagestyle{fancy}
\fancyhf{}
\lhead{Final Assignment}

\title{Computer Workshop Final Assignment}
\author{Habibollah Panbechi 401411345}

\begin{document}

\begin{titlepage}
    \maketitle
\end{titlepage}
\tableofcontents
\newpage

\section{Git and GitHub}

\subsection{Repository Initialization and Commits}
\begin{enumerate}
    \item First I created a new repository by clicking the new button on the GitHub homepage.
    \item Then created a new local repository with git on my computer by entering ``git init'' in terminal.
    \item Created a basic ``READEME.md'' and pushed it on GitHub.
    \item Cloned ``\href{https://github.com/MiliAxe/CW1402Final}{https://github.com/MiliAxe/CW1402Final}'' to use it's workflow. copied ``.github'' folder and all of it's components in my project directory. Then added it to git by using this command ``git add .github'' and pushed to GitHub again.
    \item Created a basic \LaTeX{} file called \emph{main.tex}, added to git like before the pushed again. This time I created a new tag ``v.0.0.0'' and assigned it to the lastest commit by using this command ``git tag -a v0.0.0 \verb|<|HASHOFCOMMIT\verb|>|''.
    \item Now the repository is set. Every time I finish a subsection, I commit and push it with a new tag, then GitHub automatically complies this document.
\end{enumerate}


\subsection{GitHub Actions for LaTeX Compilation}
To use a GitHub workflow you have to setup a .yml file and configure it your desires.
\begin{enumerate}
    \item Create a directory named ``.github'', by using ``mkdir .github'' command.
    \item Create another directory inside that named ``workflows''.
    \item Create a file name ``main.yml'' inside that by using ``touch main.yml''.
    \item Configure your .yml file to your desires.
\end{enumerate}
My configuration compiles my \LaTeX{} document whenever I push something with the tag *.*.* (Where * is any sequence of characters).

\newpage

\section{Exploration Tasks}
\subsection{Discovering Vim's Advanced Features}
\begin{enumerate}
    \item \textbf{Marks:} Marks in Vim allow you to quickly jump to specific locations within a file. the `m' key is used to mark the cursor's position, followed by a letter to name it. for example `ma' marks the current spot. then by pressing `a' you can easily jump to it. a mark can be deleted by ``:delmarks a'' or ``:delmarks!'' which deletes all marks.
    \item \textbf{Folding:} In every IDE you can fold a piece of code to focus your attention on something specific. This feature is available in vim. To manually create a fold, you can use visual mode to select text and then type `zf'. `zo' opens a fold, `zc' closes one, `zM' closes all folds and 'zR' opens all of them.
    \item \textbf{Persistent Undo and Redo:} By using ``:undofile'' you can browse your undo history and navigate through it. Also by adding ``set undofile'' to your .vimrc file your can undo and redo even after closing your file. Because it makes them persistent. 
\end{enumerate}


\end{document}

